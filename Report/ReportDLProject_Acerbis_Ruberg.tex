\documentclass[oneside]{book}
\usepackage[english]{babel}
\usepackage{amsmath}
\usepackage{framed}
\usepackage{graphicx}
\usepackage{caption}
\usepackage{subcaption}
\usepackage{amsfonts}
\usepackage{amssymb}
\usepackage{float}
\usepackage{xcolor}
\usepackage[toc,page]{appendix}
\usepackage{pythonhighlight}
\usepackage{filecontents}
\usepackage{graphicx}
\usepackage[english]{babel}


\begin{document}

\begin{titlepage}
\title{Project DL Course}
\author{Marco Acerbis, Nicolaas Ruberg}

\begin{Huge}
\begin{center}
Project DL Course
\end{center}
\end{Huge}

\begin{large}
\begin{minipage}{2in}
\textbf{project by:} \\
Marco Acerbis \\
Nicoolas Ruberg
\end{minipage}
\hfill
\end{large}

\end{titlepage}

\tableofcontents

\chapter*{Introduction}
\addcontentsline{toc}{chapter}{Introduction}

\chapter{Face Orientation Classification}
\section{VGG19}
VGG19

\section{MobileNet + Clustering}
MobileNet + Clustering

\section{CNN from the Scratch}
CNN from the Scratch

\chapter{Light Source Origin Classification}

\section{Technique 1}
Technique 1

\section{Technique 2}
Technique 2

\chapter{Conclusions}

\addcontentsline{toc}{chapter}{References}
\begin{thebibliography}{9}
\bibitem{Paper by professor Caio}
Dr. Caio;
\textit{Super Awesome Paper for CNNs}, University of Toronoto, March 13 2020.

\end{thebibliography}
\end{document}
